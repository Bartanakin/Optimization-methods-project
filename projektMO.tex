\documentclass[12pt, letterpaper, twoside]{article}
\usepackage[T1]{fontenc}
\usepackage[polish]{babel}
\usepackage[utf8]{inputenc}
\usepackage{lmodern}
\usepackage{amsmath}
\selectlanguage{polish}
\title{Projekt - Metody optymalizacji}
\author{}
\date{}
\begin{document}

\maketitle 

\( T(x) = a_0 + a_1x + a_2x^2 + ... +a_nx^n \)  - przybliżenie zależno{{\' s}}ci ciągu silnika od położenia na trasie - wektor \(a = [ a_0, ... a_n ]^T\) bedziemy optymalizować. \\\\
\centerline{\( f = C_T + C_F\)} \\
f - funkcja kosztu \\
\(C_T\) - koszt związany z czasem wykorzystania pojazdu \\
\(C_F\) - koszt związany ze zużyciem paliwa \\

\begin{center}
\(C_T = t_cP_T\) \\
\end{center}
\(P_T\) - warto{{\' s}}ć jednostki cxasu wykorzystania pojazdu\\
\(t_c\) - całkowity czas trwania podróży

\begin{center}
\(C_F = P_F \int_{0}^{x_c} F_c(T(x)) \,dx \) \\
\end{center}

\(P_T\) - koszt jednostki paliwa \\
\(x_c\) - długo{{\' s}}ć trasy\\
\(F_c\) - funkcja zużycia paliwa zależna siły ciągu \(T(x)\) wyrażona w \( \frac{k}{N m}\) \\
Dla uproszczenia przyjmujemy zależno{{\' s}}ć ( Zazwyczaj jest to zależno{{\' s}}ć zbliżona do liniowej) \\
\begin{center}
 \(F_c(T(x)) = F_C T(x) \) \\
\end{center}
Równanie opisujące zależno{{\' s}}ć mocy silnika od warunków zewnętrznych:\\
\begin{center}
\( T(x) = D_a(v) + D_o(x) \) \\
\end{center}
\( D_a(v) \) - zależno{{\' s}}ć oporu aerodynamicznego od prędko{{\' s}}ci \\
\( D_o(x) \) - zależno{{\' s}}ć oporów ruchu związanych z uwarunkowaniem trasy \\
Przyjmujemy zazwyczaj \(D_a(v) = Sv^2 \) dla stałej pewnej S.
Wtedy prędko{{\`s}}ć na pozycji x wynosi: \\
\begin{center}
\(v =\sqrt{\frac{T(x) - D_o(x) }{ S} } \)
\end{center}
Prędko{{\`s}}ć chwilowa jest pochodną pozycji po czasie: \\
\begin{center}
\(v =\frac{dx}{dt} \)
\end{center}
Po przekształceniach otrzymujemy równanie różniczkowe:
\begin{center}
\(\frac{dt}{dx} = \frac{\sqrt{S}}{\sqrt{T(x) - D_o(x)}} \) \\
\end{center}
I rozwiązujemy równanie:
\begin{center}
\(  t(0) + C = 0 \) \\
\end{center}
Wtedy:
\begin{center}
\(  t_c = t(x_c) + C \) \\
\end{center}
Ostatecznie optymalizujemy:
\begin{center}
\( f([a_0, ..., a_n]) = (t(x_c) + C )P_t + P_F F_C \int_{0}^{x_c}  T(x) \,dx \)\\
\end{center}
\begin{center}
Ograniczenia minimalizacji:\\
\end{center}
a) siłą wypadkowa nie może być być ujemna:
\begin{center}
\(T(x) - D_o(x) > 0\)
\end{center}
b) siła ciągu silnika nie może przekraczać pewnej maksymalnej warto{{\`s}}ci: 
\begin{center}
\(T(x) <= T_{max}\)
\end{center}
c) inne dodatkowe, których nie będę implementować, np.: prędko{{\`s}}ć maksymalna albo minimalny ciąg cilnika. \\\\
\begin{center}
Implementacja ograniczeń: \\
\end{center}
Przyjmuję model uproszczony, tzn. wybieram listę punktów \( x_i\) \(i = 1,2...n \) i pilnuję, żeby ograniczenia były spełnione dla tych punktów. \\
Model lepszy, ale trudniejszy do zaimplementowania, to szukanie za każdym razem punktów krytycznych funkcji ograniczeń względem x i sprawdzanie czy spełniają nierówno{{\`s}}ci.\\




\end{document}